\documentclass{article}
\usepackage{amsmath}    % For math symbols and formatting
\usepackage{amssymb}    % For symbols like \twoheadrightarrow
\usepackage{geometry}   % To adjust page margins
\usepackage{fancyhdr}   % For header and footer customization
\usepackage{tikz}       % For drawing diagrams

% Page layout
\geometry{a4paper, margin=1in}

% Header and Footer
\pagestyle{fancy}
\fancyhf{}
\lhead{Haskell Notes}
%\rhead{Class Notes}
\cfoot{\thepage}

% Title and Author
\title{\textbf{Haskell Notes}}
\author{Mauro Arcidiacono}
\date{November -- December 2024}

\begin{document}

\maketitle

\section*{Equivalence Relation}

A relation $R \subseteq S \times S$ is an \textbf{equivalence relation} whenever, for $s, t, u \in S$:
\begin{itemize}
    \item $R$ is \textbf{reflexive}, i.e., $(s, s) \in R$;
    \item $R$ is \textbf{symmetric}, i.e., if $(s, t) \in R$, then $(t, s) \in R$;
    \item $R$ is \textbf{transitive}, i.e., if $(s, t) \in R$ and $(t, u) \in R$, then $(s, u) \in R$.
\end{itemize}

This definition applies to various contexts in mathematics and computer science. Equivalence relations are useful in defining partitions of sets and modeling relationships like equality, congruence, or similarity.

\textbf{Define the Renaming} $[y/x]$, \textbf{in the most general way:}
\begin{itemize}
    \item regardless of the free/bound/binding position,
    \item to be applied only if $y$ does not occur in $M$.
\end{itemize}

\[
x[y/x] \equiv y
\]

\[
z[y/x] \equiv z \quad \text{if } x \neq z
\]

\[
(MN)[y/x] \equiv (M[y/x])(N[y/x])
\]

\[
(\lambda x.M)[y/x] \equiv \lambda y.(M[y/x])
\]

\[
(\lambda z.M)[y/x] \equiv \lambda z.(M[y/x]) \quad \text{if } x \neq z
\]
\section*{$\alpha$-Equivalence Relation}

Define when two lambda-terms are \textbf{"the same up to renaming of bound variables"}.

\subsection*{Definition}
\textbf{$\alpha$-equivalence}: The smallest congruence relation on $\lambda$ terms such that for all terms $M$ and all variables $y$ that do not occur in $M$:
\[
\lambda x.M =_\alpha \lambda y.(M[y/x])
\]

\subsection*{Key Properties}
\begin{itemize}
    \item \textbf{"Equivalence"}: must satisfy reflexivity, symmetry, and transitivity.
    \item It must respect the structures of lambda terms and the free/bound occurrences therein:

    \begin{itemize}
        \item If $M = M'$ and $N = N'$, then: $MN = M'N'$
        \item If $M = M'$, then: $\lambda x.M = \lambda x.M'$
        \item If $y \notin M$, then: $\lambda x.M = \lambda y.(M'[y/x])$
    \end{itemize}
\end{itemize}

\section*{$\beta$-Equivalence Relation}

\subsection*{Definition}
$\to_\beta$ is the smallest relation on terms such that:
\begin{itemize}
    \item $(\lambda x.M)N \to_\beta M[N/x]$
    \item if $M \to_\beta M'$, then $MN \to_\beta M'N$
    \item if $N \to_\beta N'$, then $MN \to_\beta MN'$
    \item if $M \to_\beta M'$, then $\lambda x.M \to_\beta \lambda x.M'$
\end{itemize}

\subsection*{$\beta$-Equivalence}
\textbf{$\beta$-equivalence} is the smallest equivalence relation, denoted by $=_\beta$, between pairs of terms, obtained by taking the reflexive, symmetric, and transitive closure of the relation $\to_\beta$.

section*{$\beta$-Equivalence and Normal Forms}

\begin{itemize}
    \item A term that has no redexes is in \textbf{normal form}.
    \item Not all terms have a normal form. For example, the term $\Omega$:
    \[
    (\lambda x.xx)(\lambda x.xx)
    \]
    has a redex, but it $\beta$-reduces to itself, without ever terminating.
    \item Other terms have different computations, all reaching the \textbf{normal form}:
\end{itemize}

\[
(\lambda x.x)((\lambda z.zz)(\lambda y.y)) \to_\beta (\lambda z.zz)(\lambda y.y) \to_\beta (\lambda y.y)(\lambda y.y) \to_\beta \lambda y.y
\]

\[
(\lambda x.x)((\lambda z.zz)(\lambda y.y)) \to_\beta (\lambda x.x)((\lambda y.y)(\lambda y.y)) \to_\beta (\lambda y.y)(\lambda y.y) \to_\beta \lambda y.y
\]

\[
(\lambda x.x)((\lambda z.zz)(\lambda y.y)) \to_\beta (\lambda x.x)((\lambda y.y)(\lambda y.y)) \to_\beta (\lambda x.x)(\lambda y.y) \to_\beta \lambda y.y
\]

\textit{... these are different reduction strategies!}

\section*{Church-Rosser Theorem and Confluence}

Assume that this multi-step reduction exists, $h > 1$, so that $M =_\beta M'$:
\[
M = M_0' \to_\beta M_1' \to_\beta \cdots \to_\beta M_h' = M'
\]

Church-Rosser theorem then asserts that there is a common reduct $N$ for $M$ and $M'$ that can be reached (with $\beta$-reductions):

\begin{figure}[ht]
    \centering
    \includegraphics[width=0.2\textwidth]{img/CR_theorem_diagram_0.jpg} % Adjust the width as needed
    \caption{Church Rosser Theorem - Confluence Example 1}
    \label{fig:cr-diagram-1}
\end{figure}

\subsection*{Key Question}
What happens when $M'$ is \emph{a} normal form?
    
    ... and when both $M'$ and $N$ are normal forms?
    
This is about \emph{THE} normal form: if one exists, it is unique.

\begin{figure}[ht]
    \centering
    \includegraphics[width=0.5\textwidth]{img/CR_theorem_diagram_1.jpg} % Adjust the width as needed
    \caption{Church Rosser Theorem - Confluence Example 2}
    \label{fig:cr-diagram-2}
\end{figure}

\subsection*{Theorem}
\textbf{Church Rosser Theorem:} For any two $\lambda$-terms $M$ and $N$, $M =_\beta N$ if and only if there is some $\lambda$-term $P$ such that:
\[
M \twoheadrightarrow_\beta^* P \quad \text{and} \quad N \xrightarrow{\beta^*} P
\]

\section*{Fixed Points in Lambda Calculus}

The Church numerals are a representation of natural numbers in the lambda calculus. They are defined as follows:
\[
\begin{aligned}
    0 &\triangleq \lambda f.\lambda x.x \\
    1 &\triangleq \lambda f.\lambda x.f(x) \\
    2 &\triangleq \lambda f.\lambda x.f(f(x)) \\
    3 &\triangleq \lambda f.\lambda x.f(f(f(x))) \\
    &\vdots
\end{aligned}
\]
The following arithmetic functions operate correctly with Church numerals:
\[
\begin{aligned}
    \text{succ} &\triangleq \lambda n.\lambda f.\lambda x.f(nfx) \quad \text{(successor)} \\
    \text{add} &\triangleq \lambda m.\lambda n.\lambda f.\lambda x.mf(nfx) \quad \text{(addition)} \\
    \text{mult} &\triangleq \lambda m.\lambda n.\lambda f.\lambda x.m(nf)x \quad \text{(multiplication)} \\
    \text{exp} &\triangleq \lambda m.\lambda n.\lambda f.\lambda x.(nm)fx \quad \text{(exponentiation)}
\end{aligned}
\]
A \textbf{fixed point} of a function $f: A \to A$ is an element $x \in A$ such that:
\[
f(x) = x
\]

Let us explore this definition by applying it to three functions on the natural numbers:
\[
\begin{aligned}
    f_1(n) &= n^2, \\
    f_2(n) &= n + 1, \\
    f_3(n) &= n.
\end{aligned}
\]

For $f_1$, the fixed points are $0$ and $1$, since $0^2 = 0$ and $1^2 = 1$. On the other hand, $f_2$ has no fixed points because no natural number satisfies $n + 1 = n$. Finally, for $f_3$, every natural number is a fixed point. This demonstrates that functions over the natural numbers can have zero, some, or all natural numbers as fixed points.

We can extend the concept of fixed points to lambda terms. A \textbf{fixed point} of a $\lambda$-term $F$ is a $\lambda$-term $X$ such that:
\[
FX =_\beta X.
\]

Interestingly, unlike functions on natural numbers, \textbf{every lambda term has at least one fixed point}. This is shown using a special $\lambda$-term called the \textbf{Y-combinator}, which generates fixed points. Recall that a combinator is a $\lambda$-term with no free variables.

The Y-combinator is defined as:
\[
Y \triangleq \lambda f.(\lambda x.f(xx))(\lambda x.f(xx)).
\]
This combinator, discovered by Curry, leads to an interesting reduction when applied to a lambda term $F$:
\[
YF \to_\beta (\lambda x.F(xx))(\lambda x.F(xx)) \to_\beta F((\lambda x.F(xx))(\lambda x.F(xx))).
\]

We also observe:
\[
F(YF) \to_\beta F((\lambda x.F(xx))(\lambda x.F(xx))).
\]

From this, we see that $YF =_\beta F(YF)$, meaning $YF$ is a fixed point of $F$. This works for any lambda term $F$, showing that the Y-combinator is a general method for finding fixed points in the lambda calculus.


\end{document}

\documentclass{article}
\usepackage{amsmath}    % For math symbols and formatting
\usepackage{geometry}   % To adjust page margins
\usepackage{fancyhdr}   % For header and footer customization

% Page layout
\geometry{a4paper, margin=1in}

% Header and Footer
\pagestyle{fancy}
\fancyhf{}
\lhead{Haskell Notes}
%\rhead{Class Notes}
\cfoot{\thepage}

% Title and Author
\title{\textbf{Haskell Notes}}
\author{Mauro Arcidiacono}
\date{November -- December 2024}

\begin{document}

\maketitle

\section*{Equivalence Relation}

A relation $R \subseteq S \times S$ is an \textbf{equivalence relation} whenever, for $s, t, u \in S$:
\begin{itemize}
    \item $R$ is \textbf{reflexive}, i.e., $(s, s) \in R$;
    \item $R$ is \textbf{symmetric}, i.e., if $(s, t) \in R$, then $(t, s) \in R$;
    \item $R$ is \textbf{transitive}, i.e., if $(s, t) \in R$ and $(t, u) \in R$, then $(s, u) \in R$.
\end{itemize}

This definition applies to various contexts in mathematics and computer science. Equivalence relations are useful in defining partitions of sets and modeling relationships like equality, congruence, or similarity.
\\
\\
\textbf{Define the Renaming} $[y/x]$, \textbf{in the most general way:}
\begin{itemize}
    \item regardless of the free/bound/binding position,
    \item to be applied only if $y$ does not occur in $M$.
\end{itemize}

\[
x[y/x] \equiv y
\]

\[
z[y/x] \equiv z \quad \text{if } x \neq z
\]

\[
(MN)[y/x] \equiv (M[y/x])(N[y/x])
\]

\[
(\lambda x.M)[y/x] \equiv \lambda y.(M[y/x])
\]

\[
(\lambda z.M)[y/x] \equiv \lambda z.(M[y/x]) \quad \text{if } x \neq z
\]
\section*{$\alpha$-Equivalence Relation}

Define when two lambda-terms are \textbf{"the same up to renaming of bound variables"}.

\subsection*{Definition}
\textbf{$\alpha$-equivalence}: The smallest congruence relation on $\lambda$ terms such that for all terms $M$ and all variables $y$ that do not occur in $M$:
\[
\lambda x.M =_\alpha \lambda y.(M[y/x])
\]

\subsection*{Key Properties}
\begin{itemize}
    \item \textbf{"Equivalence"}: must satisfy reflexivity, symmetry, and transitivity.
    \item It must respect the structures of lambda terms and the free/bound occurrences therein:

    \begin{itemize}
        \item If $M = M'$ and $N = N'$, then: $MN = M'N'$
        \item If $M = M'$, then: $\lambda x.M = \lambda x.M'$
        \item If $y \notin M$, then: $\lambda x.M = \lambda y.(M'[y/x])$
    \end{itemize}
\end{itemize}

\section*{$\beta$-Equivalence Relation}

\subsection*{Definition}
$\to_\beta$ is the smallest relation on terms such that:
\begin{itemize}
    \item $(\lambda x.M)N \to_\beta M[N/x]$
    \item if $M \to_\beta M'$, then $MN \to_\beta M'N$
    \item if $N \to_\beta N'$, then $MN \to_\beta MN'$
    \item if $M \to_\beta M'$, then $\lambda x.M \to_\beta \lambda x.M'$
\end{itemize}

\subsection*{$\beta$-Equivalence}
\textbf{$\beta$-equivalence} is the smallest equivalence relation, denoted by $=_\beta$, between pairs of terms, obtained by taking the reflexive, symmetric, and transitive closure of the relation $\to_\beta$.

section*{$\beta$-Equivalence and Normal Forms}

\begin{itemize}
    \item A term that has no redexes is in \textbf{normal form}.
    \item Not all terms have a normal form. For example, the term $\Omega$:
    \[
    (\lambda x.xx)(\lambda x.xx)
    \]
    has a redex, but it $\beta$-reduces to itself, without ever terminating.
    \item Other terms have different computations, all reaching the \textbf{normal form}:
\end{itemize}

\[
(\lambda x.x)((\lambda z.zz)(\lambda y.y)) \to_\beta (\lambda z.zz)(\lambda y.y) \to_\beta (\lambda y.y)(\lambda y.y) \to_\beta \lambda y.y
\]

\[
(\lambda x.x)((\lambda z.zz)(\lambda y.y)) \to_\beta (\lambda x.x)((\lambda y.y)(\lambda y.y)) \to_\beta (\lambda y.y)(\lambda y.y) \to_\beta \lambda y.y
\]

\[
(\lambda x.x)((\lambda z.zz)(\lambda y.y)) \to_\beta (\lambda x.x)((\lambda y.y)(\lambda y.y)) \to_\beta (\lambda x.x)(\lambda y.y) \to_\beta \lambda y.y
\]

\textit{... these are different reduction strategies!}

\end{document}

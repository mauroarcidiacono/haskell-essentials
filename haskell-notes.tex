\documentclass{article}
\usepackage{amsmath}    % For math symbols and formatting
\usepackage{geometry}   % To adjust page margins
\usepackage{fancyhdr}   % For header and footer customization

% Page layout
\geometry{a4paper, margin=1in}

% Header and Footer
\pagestyle{fancy}
\fancyhf{}
\lhead{Haskell Notes}
\rhead{Class Notes}
\cfoot{\thepage}

% Title and Author
\title{\textbf{Haskell Notes}}
\author{Mauro Arcidiacono}
\date{November -- December 2024}

\begin{document}

\maketitle

\section*{Equivalence Relation}

A relation $R \subseteq S \times S$ is an \textbf{equivalence relation} whenever, for $s, t, u \in S$:
\begin{itemize}
    \item $R$ is \textbf{reflexive}, i.e., $(s, s) \in R$;
    \item $R$ is \textbf{symmetric}, i.e., if $(s, t) \in R$, then $(t, s) \in R$;
    \item $R$ is \textbf{transitive}, i.e., if $(s, t) \in R$ and $(t, u) \in R$, then $(s, u) \in R$.
\end{itemize}

This definition applies to various contexts in mathematics and computer science. Equivalence relations are useful in defining partitions of sets and modeling relationships like equality, congruence, or similarity.

\end{document}
